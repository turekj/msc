\documentclass{report}

% language settings %
\usepackage[utf8]{inputenc}
\usepackage[MeX]{polski}

% bibliography with online sources %
\usepackage[backend=bibtex]{biblatex}
\bibliography{references.bib}

% package for generating urls %
\usepackage{hyperref}

\begin{document}

\chapter{Wstęp}

\section{Cassandra}

Apache Cassandra to darmowa, rozproszona baza danych przeznaczona do zarządzania dużymi ilościami ustrukturyzowanych danych. Projekt powstał, aby umożliwić realizację funkcjonalności przeszukiwania skrzynki odbiorczej użytkownika na portalu Facebook\footnote{Popularny portal społecznościowy dostępny pod adresem \url{http://fb.com}.}. Funkcjonalność ta nie była możliwa do zrealizowania w~oparciu o~tradycyjny, relacyjny model bazy danych. W~2008 Facebook udostępnił kod źródłowy Cassandry, która następnie (w~2010 roku) została wcielona do projektów pod opieką fundacji Apache\footnote{Fundacja wspierająca powstawanie oprogramowania o~otwartym źródle.}.\cite{casshistory} 

\printbibliography


\end{document}
