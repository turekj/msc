\documentclass{report}

% language settings %
\usepackage[utf8]{inputenc}
\usepackage[MeX]{polski}

% bibliography with online sources %
\usepackage[backend=bibtex]{biblatex}
\bibliography{references.bib}

% package for generating urls %
\usepackage{hyperref}

\begin{document}

\chapter{Wstęp}

\section{Cassandra}

Apache Cassandra to darmowa, rozproszona baza danych przeznaczona do zarządzania dużymi ilościami ustrukturyzowanych danych. Projekt powstał, aby umożliwić realizację funkcjonalności przeszukiwania skrzynki odbiorczej użytkownika na portalu Facebook\footnote{Popularny portal społecznościowy dostępny pod adresem \url{http://fb.com}.}. Funkcjonalność ta nie była możliwa do zrealizowania w~oparciu o~tradycyjny, relacyjny model bazy danych. W~2008 Facebook udostępnił kod źródłowy Cassandry, która następnie (w~2010 roku) została wcielona do projektów pod opieką fundacji Apache\footnote{Fundacja wspierająca powstawanie oprogramowania o~otwartym źródle.}.\cite{casshistory} 

Apache Cassandra jest częścią ruchu NoSQL\footnote{NoSQL - \emph{Not only SQL} (ang. nie tylko SQL).}. Chociaż brakuje oficjalnej, ustanowionej odgórnie definicji pojęcia NoSQL możemy wyróżnić zestaw cech wspólnych baz danych, które powszechnie zaliczane są do tej grupy:

\begin{itemize}
	\item Wewnętrzna reprezentacja danych w~bazach NoSQL nie opiera się na tabelach i~relacjach.
	\item Brak pełnej implementacji języka SQL. Chociaż niektóre bazy NoSQL wykorzystują język zapytań o~składni podobnej do SQL, żadna z~nich nie implementuje jej w~pełni.
	\item Bazy NoSQL posiadają elastyczny model danych. Nie wykorzystują sztywnego, narzuconego z~góry schematu.
	\item Model danych w~bazach NoSQL jest zazwyczaj ukierunkowany na łatwe klastrowanie, chociaż istnieją wyjątki od tej reguły (na przykład Neo4j\footnote{Neo4j - baza danych oparta wewnętrznie o~grafy.}). \cite{nosqldistilled}
\end{itemize}

Konsekwencją ostatniej cechy jest odejście od idei właściwości ACID. Zestaw cech ACID\footnote{ACID - skrót od \emph{Atomicity, Consistency, Isolation, Durability} (ang. atomowość, jednorodność, izolacja, trwałość)} to minimalny zbiór cech, które gwarantują realizację transakcyjności w~bazie danych. \cite{transactionconcept} Klastrowanie sprawia, że nie da się zachować spójności danych na wszystkich węzłach, stąd wymagane są inne mechanizmy zachowania poprawności danych.

\printbibliography


\end{document}
