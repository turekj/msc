\documentclass[a4paper,10pt]{report}
\usepackage[utf8]{inputenc}
\usepackage[MeX]{polski}
\usepackage{amsmath}
\usepackage{fullpage}

\title{Praca magisterska - Scrabble \\ \vspace{2mm} {\normalsize Specyfikacja wymagań i~projekt techniczny}}
\author{Jakub Turek}
\date{}

\begin{document}

\maketitle

\chapter{Projekt techniczny}

\section{Moduł wspólny}

Moduł wspólny obejmuje: 

\begin{itemize}
 \item \verb|Board| planszę do gry.
 \item \verb|Field| pole planszy do gry.
 \item \verb|Tile| płytkę.
 \item \verb|TileBag| worek z~płytkami.
 \item \verb|Rack| stojak z~płytkami.
\end{itemize}

\subsection{Plansza do gry}

Plansza do gry posiada:

\begin{itemize}
 \item Rozmiar.
 \item Zestaw pól.
\end{itemize}

Luźne uwagi:

\begin{itemize}
 \item Plansza jest iterowalna. Iterator zwraca najpierw wszystkie płytki z~pierwszego wiersza, później z~drugiego...
\end{itemize}

\subsection{Pole planszy do gry}

Pole planszy do gry posiada:

\begin{itemize}
 \item Typ pola. Dopuszczalne typy: 
	\begin{itemize}
		\item Standardowe.
		\item Startowe.
		\item Podwójna premia literowa.
		\item Podwójna premia słowna.
		\item Potrójna premia literowa.
		\item Potrójna premia słowna.
	\end{itemize}
 \item Miejsce na płytkę.
\end{itemize}

\subsection{Płytka}

Płytka posiada:

\begin{itemize}
 \item Typ płytki. Dopuszczalne typy:
	\begin{itemize}
		\item Litera.
		\item Blank.
	\end{itemize}
 \item Wartość punktową.
\end{itemize}

\subsection{Worek z~płytkami}

Worek z~płytkami posiada:

\begin{itemize}
 \item Zbiór płytek.
\end{itemize}

\subsection{Stojak z~płytkami}

Stojak z~płytkami posiada:

\begin{itemize}
 \item Maksymalną pojemność.
 \item Zbiór płytek.
\end{itemize}

\section{Moduł gry}

Moduł gry składa się z:

\begin{itemize}
 \item Zarządcy zasad gry.
 \item Zarządcy gry.
 \item Graczy.
\end{itemize}

\subsection{Zarządca zasad gry}

Zarządca zasad gry jest zestawem fabryk, które na podstawie konfiguracji tworzą:

\begin{itemize}
 \item Planszę wypełnioną polami.
 \item Worek wypełniony płytkami.
 \item Zarządcę gry wraz z~ograniczonym czasem na grę i~słownikiem wyrazów.
\end{itemize}

\subsection{Zarządca gry}

Zarządca gry odpowiada za:

\begin{itemize}
 \item Losowanie początkowego gracza.
 \item Przekazywanie sterowania pomiędzy graczami.
 \item Odmierzanie czasu przeznaczonego na rozgrywkę.
 \item Sprawdzanie legalności ruchów.
 \item Sumowanie punktów.
\end{itemize}


\end{document}
