\documentclass[a4paper,twocolumn,12pt]{article}
\usepackage[utf8]{inputenc}
\usepackage[MeX]{polski}
\usepackage{fullpage}
\usepackage[breaklinks=true]{hyperref}

\title{
 \LARGE{Projekt aplikacji mobilnej umożliwiającej umieszczenie wirtualnej grafiki w rzeczywistym położeniu} 
 \\ \vspace{2mm} 
 \large{Podstawy przetwarzania obrazów}
}
\author{Michał Aniserowicz}
\date{\today}

\begin{document}

\maketitle

\begin{abstract}
Celem artykułu jest opisanie zbioru koncepcji, które posłużą do implementacji algorytmu sztucznej inteligencji grającego w~grę Scrabble w~języku polskim. W~artykule zostały przeanalizowane i~porównane dane zawarte w~dwóch głównych słownikach wyrazów do gier dla języka polskiego, przedstawione dane statystyczne ułatwiające wprowadzanie heurystyk do algorytmu, a~także opisane metody niezbędne do wyznaczania wszystkich możliwych kombinacji ruchów w~danej turze. Autor omawia również podział rozgrywki na fazy gry i~przybliża podejście, które pozwala uzyskiwać najlepsze wyniki na każdym etapie rozgrywki.
\end{abstract}

\section*{Wstęp}


\begin{thebibliography}{9}
 \small
 \bibitem{scrabble_definition} 
  \emph{Wielki słownik ortograficzny PWN}, pod red. Edwarda Polańskiego, Wyd. 3 popr. i~uzup., Warszawa, Wydawnictwo Naukowe PWN, 2012, ISBN 978-83-01-16405-8.
 \bibitem{scrabble_word_rules}
  \emph{Zasady dopuszczalności słów} [online], Polska Federacja Scrabble [dostęp: 19 grudnia 2013], Dostępny w~Internecie: \href{http://www.pfs.org.pl/zds.php}{\nolinkurl{<http://www.pfs.org.pl/zds.php>}}.
\end{thebibliography}

\end{document}
