\documentclass[a4paper,twocolumn,12pt]{article}
\usepackage[utf8]{inputenc}
\usepackage[MeX]{polski}
\usepackage{fullpage}
\usepackage{amsmath}
\usepackage{amssymb}
\usepackage{pgfplots}
\usepackage{subfigure}
\usepackage{textcomp}

\title{
 \LARGE{Projekt aplikacji mobilnej umożliwiającej umieszczenie wirtualnej grafiki w rzeczywistym położeniu} 
 \\ \vspace{2mm} 
 \large{Podstawy przetwarzania obrazów}
}
\author{Michał Aniserowicz}
\date{\today}

\begin{document}

\maketitle

\begin{abstract}
Celem artykułu jest opisanie zbioru koncepcji, które posłużą do implementacji algorytmu sztucznej inteligencji grającego w~grę Scrabble w~języku polskim. W~artykule zostały przeanalizowane i~porównane dane zawarte w~dwóch głównych słownikach wyrazów do gier dla języka polskiego, przedstawione dane statystyczne ułatwiające wprowadzanie heurystyk do algorytmu, a~także opisane metody niezbędne do wyznaczania wszystkich możliwych kombinacji ruchów w~danej turze. Autor omawia również podział rozgrywki na fazy gry i~przybliża podejście, które pozwala uzyskiwać najlepsze wyniki na każdym etapie rozgrywki.
\end{abstract}

\section{Wstęp}



\section{Dyskretny splot macierzy}

Podstawową operacją wykorzystywaną podczas przetwarzania obrazów cyfrowych jest dyskretny splot macierzy.
Pozwala on przefiltrować obraz, tzn. uwydatnić lub ukryć niektóre jego cechy, takie jak zaszumienie.

Operacja ta przebiega w następujących krokach:

\begin{enumerate}
 \item Wybór obrazu, który zostanie poddany splotowi.
 \item Określenie:
  \begin{itemize}
   \item macierzy (tzw. maski) filtru, najczęściej o niewielkich rozmiarach, np. $3 \times 3$ lub $5 \times 5$;
   \item współczynnika normalizującego, najczęściej równego sumie wartości komórek maski.
  \end{itemize}
 \item Obliczenie wartości każdego piksela obrazu wynikowego poprzez nałożenie maski na dany piksel, tj.:
  \begin{enumerate}
   \item pomnożenie wartości piskeli obrazu wejściowego przez wartości odpowiadających komórek maski;
   \item zsumowanie otrzymanych wartości;
   \item podzielenie sumy przez współczynnik normalizujący;
   \item ustawienie wyniku jako wartości odpowiedniego piksela obrazu wynikowego.
  \end{enumerate}
\end{enumerate}

Neleży zauważyć, że wartość współczynnika normalizującego równa sumie wartości komórek maski gwarantuje wartość wyniku mieszczącą się w przedziale dopuszczalnych wartości piksela (np. $0-255$)

Przebieg operacji dyskretnego splotu został zobrazowany na Rysunku \ref{fig:convolution}.

\begin{figure}[!ht]
 \begin{center}
  \scalebox{0.25}
  {
   \includegraphics{../obrazki/filtry/splot.png}
  }
 \end{center}
 \caption{
  Operacja dyskretnego splotu macierzy.
  Źródło: \cite{stec}.
 }
 \label{fig:convolution}
\end{figure}


\subsection{Przykład - Rozmycie Gaussa} \label{sec:gauss}

Przykładem praktycznego zastosowania operacji dyskretnego splotu macierzy jest operacja Rozmycia Gaussa.
Pozwala ona wygładzić obraz, tzn. zredukować jego zaszumienie.
Odbywa się to kosztem utraty ostrości.

Przykładowe maski stosowane w tej opercaji przestawia Rysunek \ref{fig:gauss_matrices}.

\begin{figure}[!ht]
 \begin{center}
  \subfigure[]{
   $\begin{bmatrix}
     1 & 2 & 1 \\
     2 & 4 & 2 \\
     1 & 2 & 1
    \end{bmatrix}$
  }
  \subfigure[]{
   $\begin{bmatrix}
     1 & 1 & 2 & 1 & 1 \\
     1 & 2 & 4 & 2 & 1 \\
     2 & 4 & 8 & 4 & 2 \\
     1 & 2 & 4 & 2 & 1 \\
     1 & 1 & 2 & 1 & 1
    \end{bmatrix}$
  }
 \end{center}
 \caption{
  Maski stosowanie w operacji Rozmycia Gaussa:
  (a) $3 \times 3$;
  (b) $5 \times 5$.
 }
 \label{fig:gauss_matrices}
\end{figure}

Maski te mają następujące cechy:

\begin{itemize}
 \item znaczenie piksela maleje wraz z jego odległością od środka maski według funkcji Gaussa;
 \item wartość piksela wynikoweogo jest uśrednieniem wartości odpowiadającego piskela wejściowego i wartości jego sąsiadów\footnote{Filtry posiadajace tę cechę nazywa się filtrami uśredniającymi.}.
\end{itemize}

Przykład działania Rozmycia Gaussa został przedstawiony na Rysunku \ref{fig:gauss_example}.

\begin{figure}[!ht]
 \begin{center}
  \subfigure[]{\includegraphics[width=0.33\textwidth]{../obrazki/filtry/gauss_before.jpg}}
  \subfigure[]{\includegraphics[width=0.33\textwidth]{../obrazki/filtry/gauss_after.jpg}}
 \end{center}
 \caption{
  Przykad działania Rozmycia Gaussa:
  (a) obraz wejściowy;
  (b) obraz wynikowy.
  Źródło: \cite{gauss}.
 }
 \label{fig:gauss_example}
\end{figure}



\section{Detekcja krawędzi} \label{sec:edge_detection}

Operacji dyskretnego splotu macierzy używa się również w celu wykrycia krawędzi na obrazie.
W tym przypadku stosowane są filtry, które pozwalają aproksymować pochodne kierunkowe intensywności obrazu (gradienty).
Pojedyncza maska filtru wykrywa gradienty obrazu dla pojedycznego kierunku.
Powszechnie stosowane filtry, takie jak operatory Prewitt i Sobela, różnią się jedynie liczbą i rodzajem masek.

Rysunek \ref{fig:edges_example} przedstawia przykładowy wynik zastosowania filtru wykrywającego krawędzie.

\begin{figure}[!ht]
 \begin{center}
  \subfigure[]{\includegraphics[width=0.33\textwidth]{../obrazki/filtry/edges_before.jpg}}
  \subfigure[]{\includegraphics[width=0.33\textwidth]{../obrazki/filtry/edges_after.jpg}}
 \end{center}
 \caption{
  Przykładowy wynik działania filtru wykrywającego krawędzie w obrazie:
  (a) obraz wejściowy;
  (b) obraz wynikowy.
  Źródło: \cite{roberts}.
 }
 \label{fig:edges_example}
\end{figure}


\subsection{Krzyż Robertsa}

Jednym z najwcześniej opracowanych algorytmów wykrywania krawędzi jest Krzyż Robertsa
Został on zaproponowany w 1963 r. przez Lawrence’a G. Robertsa.
Wykorzystuje się w nim dwie maski, z których każda wykrywa krawędzie skierowane w innym kierunku.
Maski te zostały przedstawione na Rysunku \ref{fig:roberts_matrices}.

\begin{figure}[!ht]
 \begin{center}
  \subfigure[]{
   $\begin{bmatrix}
      0 & +1 \\
     -1 &  0
    \end{bmatrix}$
  }
  \subfigure[]{
   $\begin{bmatrix}
     +1 &  0 \\
      0 & -1
    \end{bmatrix}$
  }
 \end{center}
 \caption{
  Krzyżu Robertsa - maski:
  (a) dla kierunku 45\textdegree;
  (b) dla kierunku 135\textdegree.
 }
 \label{fig:roberts_matrices}
\end{figure}

Wynikowy obraz otrzymuje się poprzez obliczenie różnic modułów wartości odpowiadających sobie pikseli z obrazów powstałych przez zastosowanie masek.

Algorytm cechuje się niską złożonością obliczeniową.
Jest on jednak mało odporny na szum, tzn. powoduje wykrycie wielu krawędzi powstałych w wyniku zaszumienia obrazu wejściowego.


\subsection{Operator Prewitt}

Kolejnym algorytmem wykrywania krawędzi jest operator Prewitt.
Został on Zaproponowany w 1966 r. przez Judith M. Prewitt.

Wykorzystuje zbiór czterech masek, wykrywających krawędzie skierowane w czterech kierunkach (0\textdegree, 45\textdegree, 90\textdegree, 135\textdegree).
Maski te przedstawia Rysunek \ref{fig:prewitt_matrices}.

\begin{figure}[!ht]
 \begin{center}
  \subfigure[]{
   $\begin{bmatrix}
     -1 & 0 & +1 \\
     -1 & 0 & +1 \\
     -1 & 0 & +1
    \end{bmatrix}$
  }
  \subfigure[]{
   $\begin{bmatrix}
      0 & +1 & +1 \\
     -1 &  0 & +1 \\
     -1 & -1 &  0
    \end{bmatrix}$
  }
  \subfigure[]{
   $\begin{bmatrix}
     +1 & +1 & +1 \\
      0 &  0 &  0 \\
     -1 & -1 & -1
    \end{bmatrix}$
  }
  \subfigure[]{
   $\begin{bmatrix}
     +1 & +1 &  0 \\
     +1 &  0 & -1 \\
      0 & -1 & -1
    \end{bmatrix}$
  }
 \end{center}
 \caption{
  Operator Prewitt - maski:
  (a) dla kierunku 0\textdegree;
  (b) dla kierunku 45\textdegree;
  (c) dla kierunku 90\textdegree;
  (d) dla kierunku 135\textdegree.
 }
 \label{fig:prewitt_matrices}
\end{figure}

Operator Prewitt cechuje się dużo lepszą skutecznością i odpornością na szum od Krzyża Robertsa, ale jest bardziej wymagający obliczeniowo.


\subsection{Operator Sobela}

Najczęściej stosowanym filtrem wykrywającym krawędzie na obrazie jest operator Sobela.
Został on opracowany w 1968 r. przez Irwina Sobela.
Od operatora Prewitt różni się jedynie wagami komórek masek (Rysunek \ref{fig:sobel_matrices}).

\begin{figure}[!ht]
 \begin{center}
  \subfigure[]{
   $\begin{bmatrix}
     -1 & 0 & +1 \\
     -2 & 0 & +2 \\
     -1 & 0 & +1
    \end{bmatrix}$
  }
  \subfigure[]{
   $\begin{bmatrix}
      0 & +1 & +2 \\
     -1 &  0 & +1 \\
     -2 & -1 &  0
    \end{bmatrix}$
  }
  \subfigure[]{
   $\begin{bmatrix}
     +1 & +2 & +1 \\
      0 &  0 &  0 \\
     -1 & -2 & -1
    \end{bmatrix}$
  }
  \subfigure[]{
   $\begin{bmatrix}
     +2 & +1 &  0 \\
     +1 &  0 & -1 \\
      0 & -1 & -2
    \end{bmatrix}$
  }
 \end{center}
 \caption{
  Operator Sobela - maski:
  (a) dla kierunku 0\textdegree;
  (b) dla kierunku 45\textdegree;
  (c) dla kierunku 90\textdegree;
  (d) dla kierunku 135\textdegree.
 }
 \label{fig:sobel_matrices}
\end{figure}

W porównaniu z operatorem Prewitt, zastosowanie operatora Sobela powoduje otrzymanie obrazu bardziej wygładzonego.


\subsection{Operator Scharra}

Kolejnym filtrem wykrywającym krawędzie jest operator Scharra, zaproponowany w 2000 r. przez Hanno Scharra.
Od operatorów Prewitt i Sobela różni się on jedynie wagami komórek masek (Rysunek \ref{fig:scharr_matrices}).

\begin{figure}[!ht]
 \begin{center}
  \subfigure[]{
   $\begin{bmatrix}
     -3  & 0 & +3  \\
     -10 & 0 & +10 \\
     -3  & 0 & +3
    \end{bmatrix}$
  }
  \subfigure[]{
   $\begin{bmatrix}
      0  & +3 & +10 \\
     -3  &  0 & +3  \\
     -10 & -3 &  0
    \end{bmatrix}$
  }
  \subfigure[]{
   $\begin{bmatrix}
     +3 & +10 & +3 \\
      0 &  0  &  0 \\
     -3 & -10 & -3
    \end{bmatrix}$
  }
  \subfigure[]{
   $\begin{bmatrix}
     +10 & +3 &  0 \\
     +3  &  0 & -3 \\
      0  & -3 & -10
    \end{bmatrix}$
  }
 \end{center}
 \caption{
  Operator Scharra - maski:
  (a) dla kierunku 0\textdegree;
  (b) dla kierunku 45\textdegree;
  (c) dla kierunku 90\textdegree;
  (d) dla kierunku 135\textdegree.
 }
 \label{fig:scharr_matrices}
\end{figure}

W porównaniu z operatorem Sobela, operatora Scharra lepiej wykrywa kierunek krawędzi na obrazie.


\subsection{Przykłady}

Przykłady działania omówionych algorytmów przedstawia Rysunek \ref{fig:edges_comparison}.

\begin{figure}
 \begin{center}
  \subfigure[]{\includegraphics[width=0.28\textwidth]{../obrazki/filtry/init.png}}
  \subfigure[]{\includegraphics[width=0.28\textwidth]{../obrazki/filtry/roberts.png}}
  \subfigure[]{\includegraphics[width=0.28\textwidth]{../obrazki/filtry/prewitt.png}}
  \subfigure[]{\includegraphics[width=0.28\textwidth]{../obrazki/filtry/sobel.png}}
 \end{center}
 \caption{
  Przykłady działania filtrów wykrywających krawędzie na obrazie:
  (a) obraz wejściowy;
  (b) wynik działania Krzyża Robertsa;
  (c) wynik działania operatora Prewitt;
  (d) wynik działania operatora Sobela.
  Źródło: \cite{maini}.
 }
 \label{fig:edges_comparison}
\end{figure}



\section{Algorytm Canny'ego}

Najczęściej stosowanym algorytmem wykrywnia krawędzi jest obecnie algorytm Canny'ego, powstały w 1986 roku.
Jego autorem jest John F. Canny.

Algorytm ma w założeniu osiągać trzy cele:
\begin{itemize}
 \item dobra detekcja – wykrycie jak największej liczby rzeczywistych krawędzi,
 \item dobra lokalizacja – oznaczenie danej krawędzi jak najbliżej jej rzeczywistego położenia,
 \item minimalna odpowiedź – oznaczenie danej krawędzi tylko raz; brak krawędzi powstałych w wyniku zakłóceń.
\end{itemize}

Kroki algorytmu są następujące:
\begin{enumerate}
 \item Wygładzenie obrazu.
 \item Obliczenie modułu gradientu obrazu.
 \item Usunięcie niemaksymalnych pikseli.
 \item Progowanie z histerezą.
\end{enumerate}


\subsection{Wygładzenie obrazu}

Celem pierwszego kroku jest redukcja zaszumienia obrazu wejściowego.

Wygładzenie obrazu polega na zastosowaniu filtru Gaussa (patrz sekcja \ref{sec:gauss}).
Odchylenie standardowe filtru ($\sigma$) jest parametrem algorytmu - im większa wartość $\sigma$, tym mniej fałszywych krawędzi znajdzie się na obrazie wynikowym.

Przykładowy obraz otrzymany w tym kroku został przedstawiony na Rysunku \ref{fig:canny_smoothing}.

\begin{figure}[!ht]
 \begin{center}
  \subfigure[]{\includegraphics[width=0.33\textwidth]{../obrazki/canny/0_init.png}}
  \subfigure[]{\includegraphics[width=0.33\textwidth]{../obrazki/canny/1_gauss.png}}
 \end{center}
 \caption{
  Przykładowy rezultat wygładzenia obrazu ($\sigma = 3$):
  (a) obraz wejściowy;
  (b) obraz wygładzony.
  Źródło: \cite{boldak}.
 }
 \label{fig:canny_smoothing}
\end{figure}


\subsection{Obliczenie gradientu}

Kolejny krok na na celu znalezienie potencjalnych krawędzi na obrazie.
Polega on na zastosowaniu jednego z filtrów wykrywających krawędzie (patrz sekcja \ref{sec:edge_detection}).

Oprócz pozycji krawędzi, należy również zapamiętać ich kierunek.
Przykładowy wynik tego kroku algorytmu przedstawia Rysunek \ref{fig:canny_gradient}.

\begin{figure}[!ht]
 \begin{center}
  \subfigure[]{\includegraphics[width=0.33\textwidth]{../obrazki/canny/2_1_gradient.png}}
  \subfigure[]{\includegraphics[width=0.33\textwidth]{../obrazki/canny/2_2_kierunek_gradientu.png}}
 \end{center}
 \caption{
  Przykładowy rezultat drugiego kroku algorytmu Canny'ego:
  (a) potencjalne krawędzie;
  (b) kierunek gradientu (ten sam odcień szarości przypisany krawędzion o tym samym nachyleniu).
  Źródło: \cite{boldak}.
 }
 \label{fig:canny_gradient}
\end{figure}


\subsection{Usunięcie zbędnych pikseli}

Celem trzeciego kroku algorytmu jest zmniejszenie grubości krawędzi otrzymanych w poprzednim kroku tak, aby wszystkie krawędzie miały grubość jednego piksela.

Usunięcie zbędnych pikseli odbywa się poprzez porównianie każdego piksela obrazu z dwoma pikselami sąsiednimi i zachowaniu jego wartości tylko wtedy, gdy jego jasność jest większa od jasności obu sąsiadów (w przeciwnym wypadku jasność piksela jest ustawiana na 0).
Piksele sąsiednie wyznaczane są na podstawie informacji o kierunku gradientu (patrz Rysunek \ref{fig:canny_adjacent_pixels}).

\begin{figure}[!ht]
 \begin{center}
  \scalebox{0.45}
  {
   \includegraphics{../obrazki/canny/3_1_piksele.png}
  }
 \end{center}
 \caption{
  Zasada wyznaczania sąsiadów pikseli leżących na krawędzi na podstawie informacji o kierunku gradientu.
  Źródło: \cite{boldak}.
 }
 \label{fig:canny_adjacent_pixels}
\end{figure}

Przykładowy wynik tego kroku algorytmu przedstawia Rysunek \ref{fig:canny_redundant_pixels_removal}.

\begin{figure}[!ht]
 \begin{center}
  \scalebox{0.25}
  {
   \includegraphics{../obrazki/canny/3_2_krawedzie.png}
  }
 \end{center}
 \caption{
  Przykładowy rezultat usunięcia zbędnych pikseli z obrazu.
  Źródło: \cite{boldak}.
 }
 \label{fig:canny_redundant_pixels_removal}
\end{figure}


\subsection{Progowanie z histerezą}

Ostatni krok algorytmu ma na celu usunięcie zbyt ciemnych krawędzi przy zachowaniu ciemnych fragmentów jasnych krawędzi (stąd użycie słowa ``histereza''\footnote{Wyjaśnienie zjawiska histerezy dostępne pod adresem: http://pl.wikipedia.org/wiki/Histereza.}).
Przebiega on następująco:

\begin{enumerate}
 \item Określenie dwóch progów: $T_1$, $T_2$ (są to dwa kolejne, obok $\sigma$, parametry algorytmu).
 \item Zaakceptowanie krawędzi, dla których moduł gradientu jest większy od $T_2$.
 \item Usunięcie krawędzi, dla których moduł gradientu jest mniejszy od $T_1$.
 \item Rekurencyjne usunięcie krawędzi, dla których moduł gradientu jest mniejszy od $T_2$, i które nie przylegają do żadnej z już zaakceptowanych krawędzi.
\end{enumerate}


\subsection{Przykładowe rezultaty}

Przykładowe rezultaty działania algorytmu Canny'ego przedstawia Rysunek \ref{fig:canny_comparison}.

\begin{figure}[!ht]
 \begin{center}
  \subfigure[]{\includegraphics[width=0.33\textwidth]{../obrazki/canny/0_init.png}}
  \subfigure[]{\includegraphics[width=0.33\textwidth]{../obrazki/canny/4_1_wynik_3_75_125.png}}
  \subfigure[]{\includegraphics[width=0.33\textwidth]{../obrazki/canny/4_2_wynik_1_75_125.png}}
  \subfigure[]{\includegraphics[width=0.33\textwidth]{../obrazki/canny/4_3_wynik_3_25_75.png}}
 \end{center}
 \caption{
  Przykładowe wyniki algorytmu Canny'ego:
  (a) obraz wejściowy;
  (b) wynik dla $\sigma = 3, T_1 = 75, T_2 = 125$;
  (c) wynik dla $\sigma = 1, T_1 = 75, T_2 = 125$;
  (d) wynik dla $\sigma = 3, T_1 = 25, T_2 = 75$.
  Źródło: \cite{boldak}.
 }
 \label{fig:canny_comparison}
\end{figure}



\section{Podsumowanie}

\begin{itemize}
 \item Wykrywanie krawędzi (a także wiele innych operacji) sprowadza się do zastosowania dyskretnego splotu macierzy
 \item Algorytmy zaproponowane w latach 60., czy 80. są nadal stosowane
 \item Operator Sobela i algorytm Canny’ego to obecnie najpopularniejsze metody wykrywania krawędzi
\end{itemize}



\begin{thebibliography}{9}
 \small
 
 \bibitem{anal} 
  \emph{Komputerowa analiza i przetwarzanie obrazów},
  Tadeusiewicz R., Korohoda P.,
  Kraków,
  Wydawnictwo Fundacji Postępu Telekomunikacji,
  1997,
  83-86476-15-X.
 
 \bibitem{stec}
  \emph{Filtracja obrazów rastrowych} [online],
  Steć P.,
  http://www.uz.zgora.pl/$\sim$pstec/files/\\filtracja.pdf [dostęp: styczeń 2013].
 
 \bibitem{gauss}
  \emph{Rozmycie Gaussowskie} [online],
  Encyklopedia Artifice,
  http://encyklopedia.\\artifice.pl/index.php?title=Rozmycie\_\\Gaussowskie [dostęp: styczeń 2013].
 
 \bibitem{roberts}
  \emph{Krzyż Robertsa} [online],
  Wikipedia,
  http://pl.wikipedia.org/wiki/Krzyż\_\\Robertsa [dostęp: styczeń 2013].
 
 \bibitem{maini}
  \emph{Study and Comparison of Various Image Edge Detection Techniques} [online],
  Maini R., Aggarwal H.,
  http://wwwmath.tau.ac.\\il/$\sim$turkel/notes/Maini.pdf [dostęp: styczeń 2013].
 
 \bibitem{boldak}
  \emph{Wykrywanie cech w obrazach cyfrowych} [online],
  Bołdak C.,
  http://aragorn.pb.\\bialystok.pl/$\sim$boldak/DIP/CPO-W04-v01-50pr.pdf [dostęp: styczeń 2013].
  
\end{thebibliography}

\end{document}
