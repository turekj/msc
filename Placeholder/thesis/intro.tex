\chapter{Wstęp}

\section{Przedmiot pracy}

Przedmiotem niniejszej pracy magisterskiej jest aplikacja mobilna umożliwiająca umieszczenie wirtualnego obrazu w rzeczywistej lokalizacji.
Przebieg działania aplikacji przedstawia się następująco:
\begin{itemize}
 \item użytkownik wybiera plik obrazu i nakierowuje kamerę telefonu na miejsce (np. gniazdko elektryczne na ścianie), na której chce go umieścic;
 \item następnie aplikacja zapamiętuje tło obrazu (np. wspomniane gniazdko elektryczne);
 \item kiedy użytkownik ponownie wskaże dane miejsce kamerą telefonu, na ekranie urządzenia pojawi się - odpowiednio obrócony i przeskalowany - wybrany obraz.
\end{itemize}

Aplikacja umożliwia również przechowywanie danych na serwerze, tak aby umieszczone przez danego użytkownika obrazy mogły być oglądane także przez innych użytkowników.



\section{Dziedzina problemu}
Aplikacja porusza problemy zawierające się w kilku dziedzinach:

\begin{itemize}
 \item rozpoznawanie obrazu (rozpoznanie tła, na którym powinien zostać wyświetlony obraz),
 \item przetwarzanie obrazu (obracanie i skalowanie obrazu),
 \item komunikacja klient-serwer (przesyłanie obrazu i danych dotyczących jego tła),
 \item przechowywanie danych (przetrzymywanie wyżej wymienionych danych w bazie danych).
\end{itemize}


\subsection{Rozpoznawanie i przetwarzanie obrazu}

\subsection{Komunikacja klient-serwer}
Komunikacja pomiędzy klientem a serwerem może być zrealizowana przy użyciu różnych protokołów:
\begin{itemize}
 \item UDP (User Datagram Procotol)~\cite{udp}
 \item TCP (Transmission Control Protocol)~\cite{tcp}
 \item HTTP (Hypertext Transfer Protocol)~\cite{http1,http2}, jako \emph{web service}\footnote{https://pl.wikipedia.org/wiki/Usługa\_internetowa} wykorzystujący jeden z
  poniższych sposobów dostępu do danych:
  \begin{itemize}
   \item SOAP (Simple Object Access Protocol)~\cite{soap}
   \item REST (Representational State Transfer)~\cite{rest}
   \item OData (Open Data Protocol)~\cite{odata}
  \end{itemize}
\end{itemize}


\subsection{Przechowywanie danych}
Najpopularniejszym sposobem przechowywania danych w tego typu aplikacjach jest wykorzystanie bazy danych.