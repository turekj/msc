% title:
\begin{titlepage}

 \begin{tabular}{ll}
  \multirow{3}{*}{\includegraphics[scale=0.3]{figures/pw.jpg}} & POLITECHNIKA WARSZAWSKA                      \\
                                                               & Wydział Elektroniki i~Technik Informacyjnych \\
                                                               & Instytut Informatyki
 \end{tabular}
 
 \begin{flushright}
  Rok akademicki 2013/2014
 \end{flushright}

 \vspace{2cm}
 
 \begin{center}
  \LARGE PRACA DYPLOMOWA MAGISTERSKA
  
  \vspace{2cm}
  
  \large Michał Aniserowicz
  
  \vspace{2cm}
  
  \textbf{[TYTUŁ]}
 \end{center}
 
 \vspace{3cm}
 
 \hfill Praca wykonana pod kierunkiem
 
 \hfill dra inż. Jakuba Koperwasa
 
 \vspace{3cm}

 \begin{flushleft}
  \begin{minipage}{7cm}
   Ocena: \dotfill \\ \\
   \hspace*{0cm} \dotfill \\[-0.7cm]
   \begin{center}
    \small\textit{Podpis Przewodniczącego Komisji Egzaminu Dyplomowego}
   \end{center}
  \end{minipage}
 \end{flushleft}

\end{titlepage}


% biography:
\newpage
\thispagestyle{empty}

% \begin{flushright}
%  Kierunek: Informatyka \\
%  Specjalność: Inżynieria Systemów Informatycznych \\
%  Data urodzenia: 1990.02.14 \\
%  Data rozpoczęcia studiów: 2009.10.01 \\
% \end{flushright}
% 
% \vspace*{3cm}
% 
% \begin{center}
%  \textbf{\textbf{Życiorys}}
% \end{center}
% 
% \vspace{1cm}
%  
%  Urodziłem się 14.02.1990 w Białymstoku.
%  Wykształcenie podstawowe odebrałem w~latach 1997-2006 w~Publicznej Szkole Podstawowej nr~9 w~Białymstoku i~Publicznym Gimnazjum nr~2 im. 42 Pułku Piechoty w~Białymstoku.
%  W~latach 2006-2009 uczęszczałem do III Liceum Ogólnokształcącego im. K. K. Baczyńskiego w~Białymstoku.
%  Od roku 2009 jestem studentem studiów dziennych pierwszego stopnia na kierunku Informatyka na wydziale Elektroniki i~Technik Informacyjnych Politechniki Warszawskiej.
%  W~marcu 2012 roku podjąłem pracę jako programista w~firmie Fun and Mobile, gdzie po kilku miesiącach zostałem zastępcą przywódcy zespołu, do którego należę również obecnie. Moją pasją jest programowanie aplikacji w~technologii .NET Framework.
% 
% \vspace{2cm}
% 
% \begin{flushright}
%  \begin{minipage}{5cm}
%   \dotfill \\[-0.7cm]
%   \begin{center}
%   \small Podpis studenta
%   \end{center}
%  \end{minipage}
% \end{flushright}
% 
% \vspace{4cm}
% 
% \begin{flushleft}
%  Egzamin dyplomowy: \\
%  Złożył egzamin dyplomowy w dniu: \dotfill \\
%  z wynikiem: \dotfill \\
%  Ogólny wynik studiów: \dotfill \\
%  Dodatkowe uwagi i~wnioski Komisji: \dotfill \\
%  \hspace{0cm} \dotfill
% \end{flushleft}
% 
%  
% % abstract:
% \newpage
% \thispagestyle{empty}
% 
% \textbf{Streszczenie} \\
% 
%  Celem pracy jest stworzenie platformy wspomagającej tworzenie aplikacji wieloosobowych dla systemu Android, zawierającej szynę komunikacyjną klient-serwer, jak również komponenty realizujące koncepcję rzeczywistości rozszerzonej.
%  Dodatkowym celem jest zbadanie możliwości wykorzystania w~systemie Android praktyk programistycznych powszechnie stosowanych podczas pracy nad dużymi projektami.
%  Postawione cele zostały w~pełni zrealizowane - wynikiem prac jest implementacja szyny komunikacyjnej, a~także komponentów umożliwiających odczytywanie współrzędnych geograficznych urządzenia oraz wyświetlanie na jego ekranie stabilizowanej grafiki trójwymiarowej.
%  Możliwości platformy zostały zaprezentowane w~przykładowej grze wieloosobowej.
%  Praca zawiera opis działania wykonanych komponentów, przedstawia proces ich projektowania, a~także wyjaśnia ich wewnętrzną strukturę.
% 
% \vspace*{\stretch{1}}
% 
% \begin{center}
%  \large \textbf{Client-server Augmented Reality applications framework for Android system}
% \end{center}
% 
% \vspace*{1cm}
% 
% \textbf{Summary} \\
% 
%  The goal of this thesis is to create a~framework supporting the development of multiuser applications for Android system.
%  The framework should consist of a~client-server bus, as well as several Augmented Reality components.
%  An additional goal is to explore the possibility of adapting commonly used programming practises dedicated to large projects during Andorid applications development.
%  All the goals have been fully accoplished - the output is the implementation of the client-server bus as well as AR components allowing to track the device's geographic coordinates and to render a~stable three-dimensional graphics on the display of the device.
%  In order to demonstrate the features of the framework, a~sample multiplayer game has been created.
%  The thesis includes a~description of the created components' design process, as well as their functionality and structure.
% 
% \vspace*{\stretch{1}}